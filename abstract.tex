\documentclass{article}

\begin{document}

\title{Evanescent-wave Johnson noise from BCS superconductors}


\begin{abstract}
Qubits near conducting devices are susceptible to decoherence and heating electromagnetic field fluctuations, which leak from the devices as evanescent-wave Johnson noise.
This noise depends on the electromagnetic response function of the device material.
For BCS superconductors, this response function has been well studied for momenta near the Fermi surface, at arbitrary impurities.
We provide a method for estimating this response function for BCS conductors at high momenta by interpolating with the normal state.
This response function is used to calculate the noise outside the surface of a superconducting half-space.
We present a frequency-temperature relationship that categorises the noise's transition from normal-state to superconducting behaviour. 
Additionally, we find that, unlike in the normal state, surface waves contribute significantly to the total noise.
Lastly, we discuss an extension for nonequilibrium superconductors.
These serve as considerations for qubit device architectures, as well as guide for using charge or spin qubits as probes of superconducting devices.
\end{abstract}



\end{document}
